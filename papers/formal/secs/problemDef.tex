
\section{Problem Statement}
From one perspective, \textit{Approximate Programming by Example} (\approximatePBE) can be seen as a generalization of classic Programming by Example (PBE).
Specifically, \approximatePBE relaxes the correctness criteria of PBE.
The problem statement of PBE is roughly to find a program that correctly maps every given input-output examples, and generalizes this mapping in a reasonable way across the input space, for a definition of reasonable that varies across application domains and search algorithms.
In contrast, \approximatePBE searches for a program that only \textit{closely} maps every given input-output example, and again generalizes the mapping across the input space.
The key difference is that in \approximatePBE, we do not require the examples must be mapped exactly.

Note that this is distinct from a PBE setting with noisy examples - we cannot throw away examples that are anomalous and learn over the remaining examples~\cite{raychev2016learning}.
In \approximatePBE we want to find a program that fits as closely as possible to \textit{all} provided examples.
We additionally distinguish this problem from a general supervised learning problem, since in \approximatePBE, we have an additional constraint that the generated model must be a readable program.
We then formally state \approximatePBE as the following optimization problem.

\noindent\textbf{Given:}
\begin{itemize}[topsep=0pt]
  \item A domain of input-output examples $\exampleDomain$, and a distance function $d:\exampleDomain \to \exampleDomain \to \reals$ such that $(\exampleDomain,\distFxn)$ is a metric space.
  \item A user specified threshold $\epsilon \in \reals$ for minimal distance with respect to the metric space.
  \item A set of input,output examples $\{(i_0,o_0),...,(i_n,o_n)\}$ where $\forall 0 \leq j \leq n.\ i_j,o_j \in \exampleDomain$.
  \item A grammar $G$ of programs over the input-output example domain. 
        We notate the space of programs that can be generated from a grammar $G$ as $\languageOf{G}$.
        The grammar $G$ generates programs structures, that must be completed with constants $\constants$ - $\forall p \in \languageOf{G}.\ f: \constants \to \exampleDomain \to \exampleDomain$. 
        The program structure can be made into an function over the example space by fixing the constants, $\constants$, so that partial application of the program yields, $p(\constants): \exampleDomain \to \exampleDomain$. 
        The type of $\constants$ should be over a continuous space, for example $\reals^n$. 
  \item A cost function $\costFxn : \constants \to \languageOf{G} \to (\exampleDomain,\exampleDomain)^n \to \reals$ that calculates how well a particular $p(\constants) \in \languageOf{G}$ maps the $n$ input examples to the corresponding $n$ output examples. This is a user-defined error aggregation function of distance function, $\distFxn$, for example in $\costFxn(o,i,p(\constants)) = \sum_{j=0}^{n} (\distFxn(o_j,p(\constants,i_j)))$.
\end{itemize}
\textbf{Minimize:}

Find a program $p \in \languageOf{G}$ and constants $\constants$ to minimize $\costFxn(o,i,p(\constants))$. 
With respect to practical synthesis, find $p$ and $k$ such that $\costFxn(o,i,p(\constants)) \leq \epsilon$.
\vspace{\baselineskip}

There are a number of existing tools and techniques to solve general minimization/optimization problems over metric spaces~\cite{optmizationTextbook}.
However, one of the core components of such optimizations is the need for repeated computation of the cost function $\costFxn$ with candidate solutions, $p \in \languageOf{G}$.
When $c$ is particularly slow to compute, for instance when $\distFxn$ uses blackbox I/O, exploring the complete space of $\languageOf{G}$ is prohibitively expensive.
To overcome this challenge, we can proactively prune the search space before applying optimization techniques.
Since the optimization problem of \approximatePBE is over a space of programs, we can leverage techniques from formal methods to first refine the search space by generating a $G'$ such that $\languageOf{G'} \subset \languageOf{G}$.

\begin{exmp}
Applying \approximatePBE to Audio DSP.

In the domain of Audio Digital Signal Processing, the \exampleDomain is audio waveforms of type \texttt{[Double]}.
We will use as the distance function $\distFxn$ an audio fingerprinting technique~\cite{SantolucitoFARM} that measures how different two audio clips sound.

We define our grammar $G$ for a DSP program as follows:

\begin{align*}
DSP &= P \arrComp P\ |\ P \\
P   & = LPF \ [0,1]^2 \ |\ HPF \ [0,1]^2 \ |\ WN \ [0,1]
\end{align*}

For a program $p \in \languageOf{G}$ generated from this grammar, we must fix constants for each parameterized node of the AST.
In this case, we must then choose $\constants :: \reals^{5}$ to generate an executable $p:\exampleDomain \to \exampleDomain$. 

\end{exmp}
