%% For double-blind review submission
%\documentclass[sigplan,screen]{acmart}\settopmatter{printfolios=true}

%% For single-blind review submission
%\documentclass[acmlarge,review]{acmart}\settopmatter{printfolios=true}
%% For final camera-ready submission
%\documentclass[sigplan,screen]{acmart}\settopmatter{}

\documentclass[sigplan,10pt,review,anonymous]{acmart}\settopmatter{printfolios=true,printccs=false,printacmref=false}

%% Note: Authors migrating a paper from PACMPL format to traditional
%% SIGPLAN proceedings format should change 'acmlarge' to
%% 'sigplan,10pt'.

\usepackage{etex}

%% Some recommended packages.
                        %% http://ctan.org/pkg/booktabs
\usepackage{subcaption} %% For complex figures with subfigures/subcaptions
                        %% http://ctan.org/pkg/subcaption


\newcommand{\cL}{{\cal L}}
\let\terms\undefined

\usepackage{balance}

\long\def\markk#1{{\color{red}{\bf Mark: }{\small [#1]}}}



%\makeatletter\if@ACM@journal\makeatother
%% Journal information (used by PACMPL format)
%% Supplied to authors by publisher for camera-ready submission

%% Copyright information
%% Supplied to authors (based on authors' rights management selection;
%% see authors.acm.org) by publisher for camera-ready submission
%\setcopyright{acmcopyright}
%\setcopyright{acmlicensed}
%\setcopyright{rightsretained}
%\copyrightyear{2017}           %% If different from \acmYear

\copyrightyear{2018} 
\acmYear{2018} 
\setcopyright{acmlicensed}
\acmConference[FARM '18]{Proceedings of the 6th ACM SIGPLAN International Workshop on Functional Art, Music, Modeling, and Design}{September 29, 2018}{St. Louis, MO, USA}
\acmBooktitle{Proceedings of the 6th ACM SIGPLAN International Workshop on Functional Art, Music, Modeling, and Design (FARM '18), September 29, 2018, St. Louis, MO, USA}
\acmPrice{15.00}
\acmDOI{10.1145/3242903.3242906}
\acmISBN{978-1-4503-5856-9/18/09}



%% Bibliography style
\bibliographystyle{ACM-Reference-Format}
%% Citation style
%% Note: author/year citations are required for papers published as an
%% issue of PACMPL.
\citestyle{acmauthoryear}   %% For author/year citations



\begin{document}

%% Title information
\title[DSP-PBE for Noncommutative Filters]{Digital Signal Processing Programming by Example for Noncommutative Filters}
                                        %% [Short Title] is optional;
                                        %% when present, will be used in
                                        %% header instead of Full Title.
%\titlenote{with title note}             %% \titlenote is optional;
                                        %% can be repeated if necessary;
                                        %% contents suppressed with 'anonymous'
%\subtitle{Subtitle}                     %% \subtitle is optional
%\subtitlenote{with subtitle note}       %% \subtitlenote is optional;
                                        %% can be repeated if necessary;
                                        %% contents suppressed with 'anonymous'

%% Author information
%% Contents and number of authors suppressed with 'anonymous'.
%% Each author should be introduced by \author, followed by
%% \authornote (optional), \orcid (optional), \affiliation, and
%% \email.
%% An author may have multiple affiliations and/or emails; repeat the
%% appropriate command.
%% Many elements are not rendered, but should be provided for metadata
%% extraction tools.

%% Author with single affiliation.
\author{Mark Santolucito}
\orcid{0000-0001-8646-4364}             %% \orcid is optional
\affiliation{
  \department{Computer Science}              %% \department is recommended
  \institution{Yale University}            %% \institution is required
  \streetaddress{51 Prospect St.}
  \city{New Haven}
  \state{CT}
  \postcode{06511}
  \country{USA}
}
\email{mark.santolucito@yale.edu}          %% \email is recommended

\author{Ruzica Piskac}
\affiliation{
  \department{Computer Science}              %% \department is recommended
  \institution{Yale University}            %% \institution is required
  \streetaddress{51 Prospect St.}
  \city{New Haven}
  \state{CT}
  \postcode{06511}
  \country{USA}
}
\email{ruzica.piskac@yale.edu}          %% \email is recommended


%\titlenote{This research sponsored by NSF grants CCF-1302327 and CCF-1715387.}




%% Paper note
%% The \thanks command may be used to create a "paper note" ---
%% similar to a title note or an author note, but not explicitly
%% associated with a particular element.  It will appear immediately
%% above the permission/copyright statement.
%\thanks{Many thanks to Thomas Murphy for his guidance in thinking through this problem and many solution attempts. Thanks to the anonymous reviewers whose comments helped raise the bar on this paper.}       %% \thanks is optional
                                        %% can be repeated if necesary
                                        %% contents suppressed with 'anonymous'


%% Abstract
%% Note: \begin{abstract}...\end{abstract} environment must come
%% before \maketitle command
\begin{abstract}
Digital Signal Processing (DSP) programming requires users to have experience with both programming fundamentals and an understanding of digital audio.
To assist novice DSP programmers in writing code, we introduce a tool for automatically generating SuperCollider programs from examples of audio input and output.
A user provides an example of an audio transformation in the form of two audio files, an input file and an output file.
The system takes these two audio files and generates a SuperCollider program that transforms the input to the output.
In order to support the synthesis of DSP programs on complex filters such as noncommutative filters, we introduce a new synthesis algorithm that searches for a structural form of a filter and the parameter values for the filter in two separate stages.
We present the results of running our tool on a set of benchmark audio example pairs.

\end{abstract}


%% 2012 ACM Computing Classification System (CSS) concepts
%% Generate at 'http://dl.acm.org/ccs/ccs.cfm'.
\begin{CCSXML}
  <ccs2012>
  <concept>
  <concept_id>10010405.10010469.10010475</concept_id>
  <concept_desc>Applied computing~Sound and music computing</concept_desc>
  <concept_significance>500</concept_significance>
  </concept>
  </ccs2012>
\end{CCSXML}

\ccsdesc[500]{Applied computing~Sound and music computing}

\maketitle



\section{Introduction}

There has been a proliferation of new programming languages for audio Digital Signal Processing (DSP) with languages such as SuperCollider, Faust, MaxMSP, PureData.
These languages provide a high-level interface to make DSP programming more approachable for new-comers to the field of audio programming.
DSP programming poses a particular challenge to novice programmers as writing a DSP program requires an understanding of traditional programming concepts such as loops and conditionals, as well as the ability to reason about DSP issues such as time vs frequency domain.
To assist beginning DSP programmers we turn to the Programming by Example paradigm.

Programming by Example (PBE) is a program synthesis technique that allows users to provide input and output examples to system that then automatically generates code that models the illustrated functionality.
This technique has found particular success in spreadsheets with the FlashFill tool.
In a similar vein, we hope to make DSP programming more accessible to a larger audience by using Digital Signal Processing Programming by Example (DSP-PBE).

The goal of DSP-PBE is to take an input and output audio example from the user, and synthesize the DSP program, $F$, that minimizes the distance between the transformed input, $F(i)$, and the output $o$.
In this way, a user can automatically generate the program code that captures the audio transformation demonstrated by the provided examples.
A key part of this technique is that the user receives readable DSP program code that can be further tuned or edited as the user sees fit, opening the door for learning opportunities and creative invention.


\subsection{Motivating Example}


\begin{figure}
\begin{lstlisting}
( 
SynthDef(\dsp_pbe, {|out=0|

   var main_in, id7, out6, lpf5, hpf4, psh1;
   main_in = PlayBuf.ar(2, ~buf);
   psh1 = FreqShift.ar(pitchRatio: -399.999, 
                       mul: 0.55, 
                       in: In.ar());
   hpf4 = HPF.ar(freq: 10100.0, 
                 mul: 1.562e-10, 
                 in: psh1);
   lpf5 = LPF.ar(freq: 3860.002, 
                 mul: 0.85, 
                 in: psh1);
   out6 = Mix.ar(2, [hpf4, lpf5]);
   id7 = 0.7 * out6;
   Out.ar(out, id7);

}).add;
)
\end{lstlisting}
\caption{The SuperCollider program synthesized by \ourTool to simulate the effect of a trumpet hat mute.}
\label{fig:sc_code}
\end{figure}

We demonstrate the application of \ourTool by showing how the tool is used to build a SuperCollider program that mimics the effect of a trumpet mute.
A user may want to construct a program in order to apply the effect of a trumpet mute to another sound.
To start, a user would provide an audio input example file of the trumpet without a mute (``00 none'' in Table~\ref{table:eval}), and an audio output example file (``01 hat'' in Table~\ref{table:eval}) of the trumpet playing the same note with the mute.
The user then invokes \ourTool on these two files, and the tool generates the SuperCollider program shown in Fig.~\ref{fig:sc_code}.
With this code, the user can then use the program directly in a larger SuperCollider project.
In addition, the user may wish to edit the code themselves - for example changing the mul argument to the HPF to a larger value.

In summary, this paper makes the following contributions.

\begin{enumerate}
\item We propose framework for the synthesis of DSP programs that utilize noncommutative filter types 
\item We propose an algorithm for search through the possible structural forms of a DSP filter program
\item We present our tool, \ourTool, which synthesizes SuperCollider programs from audio input/output examples, and evaluate \ourTool over a set of benchmarks
\end{enumerate}





\subsection{Problem Statement}
At a high level, \textit{Approximate Programming by Example} (\approximatePBE) can be seen as a generalization of classic Programming by Example (PBE).
Specifically, \approximatePBE relaxes the correctness criteria of PBE.
Whereas PBE searches for a program that correctly maps every given input-output examples, \approximatePBE searches for a program that closely maps every given input-output example.
We then formally state \approximatePBE as the following optimization problem.

\noindent\textbf{Given:}
\begin{itemize}[topsep=0pt]
  \item A domain of input-output examples $\exampleDomain$, and a distance function $d:\exampleDomain \to \exampleDomain \to \reals$ such that $(\exampleDomain,\distFxn)$ is a metric space.
  \item A user specified threshold $\epsilon \in \reals$ for minimal distance with respect to the metric space.
  \item A set of input,output examples $\{(i_0,o_0),...,(i_n,o_n)\}$ where $\forall 0 \leq j \leq n.\ i_j,o_j \in \exampleDomain$.
  \item A grammar $G$ of programs over the input-output example domain, $\forall p \in \languageOf{G}.\ f: \constants \to \exampleDomain \to \exampleDomain$. 
    The program structure can be made into an executable function by fixing the constants, $\constants$, with partial application on the program, $p(\constants): \exampleDomain \to \exampleDomain$. We leave the type of $k$ open for now. We notate the space of programs with fixed constants that can be generated from a grammar $G$ as $\interp{\languageOf{G}}$.
  \item A cost function $\costFxn : \exampleDomain \to \exampleDomain \to \interp{\languageOf{G}} \to \reals$ that calculates how well a particular $p(\constants) \in \interp{\languageOf{G}}$ maps the input examples to the output examples. This is user-defined, but will generally utilize the distance function, $\distFxn$, for example in $\costFxn(o,i,p(\constants)) = \sum_{j=0}^{n} (\distFxn(o_j,p(\constants,i_j)))$.
\end{itemize}
\textbf{Minimize:}

Find a program $p \in \languageOf{G}$ and constants $\constants$ to minimize $\costFxn(o,i,p(\constants))$. 
With respect to practical synthesis, find $p$ and $k$ such that $\costFxn(o,i,p(\constants)) \leq \epsilon$.
\vspace{\baselineskip}

There are a number of existing tools and techniques to solve general minimization/optimization problems over metric spaces~\cite{optmizationTextbook}.
However, one of the core components of such optimizations is the need for repeated computation of the cost function $\costFxn$ with candidate solutions, $p \in \languageOf{G}$.
When $c$ is particularly slow to compute, for instance when $\distFxn$ uses blackbox I/O, exploring the complete space of $\languageOf{G}$ is prohibitively expensive.
To overcome this challenge, we can proactively prune the search space before applying optimization techniques.
Since the optimization problem of \approximatePBE is over a space of programs, we can leverage techniques from formal methods to first refine the search space by generating a $G'$ such that $\languageOf{G'} \subset \languageOf{G}$.


\section{Overview}

The core of our synthesis algorithm operates on a two layer feedback loop that incrementally refines both the possible structure and the constants of the function, as illustrated in Fig.~\ref{fig:overview}.
The algorithm terminates when, for input $i$ and output $o$, the synthesized function maps the input such that $\distFxn(\synthedFilter(i),o)$ is below a user-provided given minimum threshold $\epsilon$.
When metrical synthesis fails to produce a solution that is below the $\epsilon$ threshold, we restart synthesis by searching for a new structure.
After generating a new structure, we apply the metrical values from the previous metrical synthesis attempt to the new structure, and begin metrical synthesis again.

\section{Overview}

The core of our synthesis algorithm operates on a two layer feedback loop that incrementally refines both the possible structure and the constants of the function, as illustrated in Fig.~\ref{fig:overview}.
The algorithm terminates when, for input $i$ and output $o$, the synthesized function maps the input such that $\distFxn(\synthedFilter(i),o)$ is below a user-provided given minimum threshold $\epsilon$.
When metrical synthesis fails to produce a solution that is below the $\epsilon$ threshold, we restart synthesis by searching for a new structure.
After generating a new structure, we apply the metrical values from the previous metrical synthesis attempt to the new structure, and begin metrical synthesis again.

\section{Overview}

The core of our synthesis algorithm operates on a two layer feedback loop that incrementally refines both the possible structure and the constants of the function, as illustrated in Fig.~\ref{fig:overview}.
The algorithm terminates when, for input $i$ and output $o$, the synthesized function maps the input such that $\distFxn(\synthedFilter(i),o)$ is below a user-provided given minimum threshold $\epsilon$.
When metrical synthesis fails to produce a solution that is below the $\epsilon$ threshold, we restart synthesis by searching for a new structure.
After generating a new structure, we apply the metrical values from the previous metrical synthesis attempt to the new structure, and begin metrical synthesis again.

\input{figs/overview}






\section{Structural Synthesis}
\label{sec:struct}

The structural synthesis stage of our algorithm consists of two parts.
First we guess at an initial filter structure based on a preliminary analysis of the input/output audio examples.
In the second stage, which occurs during the synthesis loop, we iteratively pick new structures to try during metrical synthesis.
For this second stage we have implemented a greedy algorithm to pick the best structure out of the possible next choices.

\subsection{Initial Structure Construction}
\label{sec:initStruct}
In order to find an initial structure for our synthesized filter, we use an adaption of room impulse response measurement.
When measuring the impulse response of a room, we can reconstruct the band-pass filter exactly by playing a sound in the room, recording the sound, then examining the differences.
In DSP-PBE, we do not want to allow infinitely many bandpass filters, as the synthesized program should be relatively small and human-readable, as code that might have been manually written.
Thus, we mimic the room impulse response measurement technique using only the available \dspnode in our grammar.

In the grammar listed in Fig.~\ref{fig:grammar}, we have two filters which are similar to the bandpass filters used in measuring room impulse response - a low-pass filter, $LFP \ [0,20k]\ [0,1]$, and a high-pass filter $HPF\ [0,20k]\ [0,1]$.
In order to quickly discover approximate values for each \dspnode, we run an analysis of the frequencies present in the input example that have decreased in amplitude in the output example.
To analyze the inputs and 
We then build a relation from this observation.
To formalize this approach, we write a formula that describes that how to find an initial threshold value for a lowpass filter in our synthesized code.
We use $t$ to represents the threshold of the lowpass filter, \texttt{spectrogram} to represent a function that runs a FFT on a sound sample and returns a list of peaks, $f_i$ to represent a frequency, and \texttt{amp()} to represent a function to retrieve the amplitude of the frequency. 
%
\begin{align*}
&\text{Given input audio }i\text{ and }o\text{, find }t\text{ such that} \\
&\forall f_1 \in  \texttt{spectrogram(i)}.\ \forall f_2 \in \texttt{spectrogram(o)}. \\
&(f_1 > t \land  f_2 > t \land f_1 == f_2) \implies \texttt{amp}(f_1) > \texttt{amp}(f_2)
\end{align*}


Similarly, we can build a formula to describe how to calculate an initial value for a high-pass.
Here we look for the lowest frequency where amplitude has decreased in the output audio - this is then a starting point for the threshold of a high pass filter.
%
\begin{align*}
&\text{Given input audio }i\text{ and }o\text{, find }t\text{ such that} \\
&\forall f_1 \in \texttt{spectrogram(i)}.\ \forall f_2 \in \texttt{spectrogram(o)}. \\
&(f_1 < t \land f_2 < t \land f_1 == f_2) \implies \texttt{amp(}f_1\texttt{)} > \texttt{amp(} f_2 \texttt{)} 
\end{align*}


\subsection{Structural Synthesis}

\begin{figure}
\begin{align*}
	LPF \ 10000 \ 0.5 \arrComp HPF \ 100 \ 0.5 & \\
	\ldots & \arrComp PitchShift \ 0.1 \ 0.1 \\
	\ldots & \parallelCompose PitchShift \ 0.1 \ 0.1 \\
	\ldots & \arrComp Reverb \ 0.1 \ 0 \ 0.1 \\
	\ldots & \parallelCompose Reverb \ 0.1 \ 0 \ 0.1 \\
	\ldots & \arrComp WhiteNoise \ 0.1 \\
	\ldots & \parallelCompose WhiteNoise \ 0.1
\end{align*}
\caption{New structural candidates are generated based on all compositions of unused \dspnode}
\label{fig:generation}
\end{figure}

During each loop of our synthesis procedure, we attempt to build a new structure based on the results of our previous attempts.
We compare several different variations on the current structure (Fig.~\ref{fig:generation}).
Variations are generated by composing all \dspnode that are not present in the current structure with sequential and parallel composition onto the existing filter program returned from our metrical synthesis (cf. Sec.~\ref{sec:opt}).
We keep the parameters for the existing \dspnode, but for the new \dspnode we initialize the parameters to small values.
Using small values (instead of zeroes) ensures that the filter has an observable effect on the sound.
We score each of these filter programs, $f$, by the distance, $\distFxn(f(i), o)$, from their output to the output example file.
We then select the candidate filter program with the best score.
We continue this process until we either exceed our maximum allowed structural attempts or have reached a program whose output is close enough to the desired output.

\section{Metrical Synthesis}
\label{sec:opt}

In the parameter tuning phase of the algorithm we can leverage existing optimization techniques to find the best constants $\constants$ for a given program structure $p$.
One such optimization technique is gradient descent.
If the $\distFxn$ is not differentiable, for example using I/O, gradient descent may not be an appropriate option.
This stage can use any parameter tuning approach in place of gradient descent.

\subsection{Deriving Metric weights}
Additionally, we can improve our next parameter search over a new program structure by learning from synthesis attempts over other program structures.
To do this, we map the parameters from the best program in the previous parameter search onto the new program structure as an initial starting point for metrical synthesis.

This is where we are working in a numerical-driven supervised learning style.

\begin{exmp}
We take the same log from above, and see that, across all structures, \texttt{HPF 800} is slightly better than \texttt{200}, 
  and \texttt{LPF 200} is slightly better than \texttt{800}.
To leverage this new knowledge, we can accordingly adjust our initial values for the starting point of our gradient descent.
\end{exmp}

\section{Learning from failed attempts}
\label{sec:feedback}


\subsection{Deriving Structural Constraints}
One key feature of our synthesis is the feedback loop structure that allows us to learn from incorrect program structures.
Each time that the parameter tuning phase fails to produce a program $p(c)$ that meets the cost threshold, we can extract information from that search which allows us to improve our choice of program structure from within the language of the current grammar, $\languageOf{G}$.


\begin{exmp}
Take as an example a synthesis attempt that produced the following log of candidate programs and associated costs.

\begin{lstlisting}
(LPF 800 $\arrComp$ HPF 200, 80)
(LPF 200 $\arrComp$ HPF 800, 83)
(HPF 800 $\arrComp$ LPF 200, 20)
(HPF 200 $\arrComp$ LPF 800, 23)
\end{lstlisting}

From this we want to learn something about the discrete space of program structure. 
This is where we are working in a semantics-driven programming-by-example style
We can see that the form \texttt{LPF \_ >> HPF \_} is way worse than \texttt{HPF \_ >> LPF \_}.
From this, we can learn a structural constraint that we should not allow the form \texttt{LPF \_ >> HPF \_}.
We will add this as a \textit{derived structural constraint}, and continue synthesis.
\end{exmp}

\subsection{Deriving Metric Constraints}
Additionally, we can improve our next parameter search over a new program structure.
To do this, we map the parameters from the best program in the previous parameter search onto the new program structure.
In this way, we can choose a better starting point for our parameter search.

This is where we are working in a numerical-driven supervised learning style.
We take the same log from above, and see that, across all structures, \texttt{HPF 800} is slightly better than \texttt{200}, 
  and \texttt{LPF 200} is slightly better than \texttt{800}.
To leverage this new knowledge, we can accordingly adjust our initial values for the starting point of our gradient descent.


\section{Evaluation}

To evaluate our approach, we apply our synthesis algorithm to two sets of input/output examples.
The first set captures the use case of \ourTool where a user wants to model real world sounds.
This set consists of trumpet sounds with various mutes.
The second set of example input/output we run \ourTool on focuses on more creative applications.
Here we attempt to transform whitenoise into various clips from Brian Eno's Music for Airports.

The results of running \ourTool on the trumpet mute sounds benchmarks are reported in Table~\ref{table:eval}.

We need to normalize all files as the distance metric is amplitude analysis.

\begin{table}
\begin{tabular}{|l|l|c|c|c|}
\hline
\textbf{Input} & \textbf{Filter} & \textbf{dist(o, $f$(i))} & \textbf{Time (sec)}
\csvreader{results/farm.csv}{}
{\\ \hline \csvcoli & \csvcolii & \csvcoliv & \csvcolvi}
\\ \hline
\end{tabular}
\caption{Demonstrating the accuracy of the user provided refinements for the initial filter.}
\label{table:evalInit}
\end{table}

\begin{table*}[]
\begin{tabular}{|l|l|c|c|c|}
\hline
\textbf{Input} & \textbf{Output} & \textbf{dist(o, $f$(i))} & \textbf{Structural Attempts} & \textbf{Time (sec)}
\csvreader{results/trumpet.csv}{}
{\\ \hline \csvcoli & \csvcolii & \csvcoliv & \csvcolv & \csvcolvi}
\\ \hline
\end{tabular}
\caption{Evaluation on a set of benchmarks.}
\label{table:eval}
\end{table*}






\section{Related}

Previously, the machine learning and semantic modelling approach to program synthesis have been viewed as at odds with one another.
Depending on the context, one or the other outperforms the other~\cite{devlin2017robustfill}.
In this work, we join the two approaches together in a restricted example domain (where the example domain forms a metric space).

The work of~\cite{misailovic2011probabilistically} introduces a notion of approximate correctness of program transformation over probabilistic programs.
Aside from our work focusing on synthesis, we also are working over deterministic programs - our notion of approximate is not with respect to probabilistic outcomes, but closeness in the metric space.


\section{Conclusions}

\markk{this is really just filler - maybe someone else can write something more inspired}
In this paper, we have outlined an approach for learning digital models of analog effects.
We have implemented this approach in our tool, which is open-source and freely available online\footnote{\url{https://github.com/yale-omi/DSP-PBE}}.
The combination of programming-by-example and computer music is still a new direction, this work serves as a demonstration of the practical utility of this research synergy.


%\balance
\bibliography{biblio}  

%%% Place this command where you want to balance the columns on the last page. 
%\balancecolumns 

% That's all folks!
\end{document}
