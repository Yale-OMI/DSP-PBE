\section{Structural Synthesis}
\label{sec:struct}

The structural synthesis stage of our algorithm consists of two parts.
First we guess at an initial filter structure based on a preliminary analysis of the input/output audio examples.
In the second stage, which occurs during the synthesis loop, we iteratively pick new structures to try during metrical synthesis.
For this second stage we have implemented a greedy algorithm to pick the best structure out of the possible next choices.

\subsection{Initial Structure Construction}
\label{sec:initStruct}
In order to find an initial structure for our synthesized filter, we use an adaption of room impulse response measurement.
When measuring the impulse response of a room, we can reconstruct the band-pass filter exactly by playing a sound in the room, recording the sound, then examining the differences.
In DSP-PBE, we do not want to allow infinitely many bandpass filters, as the synthesized program should be relatively small and human-readable, as code that might have been manually written.
Thus, we mimic the room impulse response measurement technique using only the available \dspnode in our grammar.

In the grammar listed in Fig.~\ref{fig:grammar}, we have two filters which are similar to the bandpass filters used in measuring room impulse response - a low-pass filter, $LFP \ [0,20k]\ [0,1]$, and a high-pass filter $HPF\ [0,20k]\ [0,1]$.
In order to quickly discover approximate values for each \dspnode, we run an analysis of the frequencies present in the input example that have decreased in amplitude in the output example.
To analyze the inputs and 
We then build a relation from this observation.
To formalize this approach, we write a formula that describes that how to find an initial threshold value for a lowpass filter in our synthesized code.
We use $t$ to represents the threshold of the lowpass filter, \texttt{spectrogram} to represent a function that runs a FFT on a sound sample and returns a list of peaks, $f_i$ to represent a frequency, and \texttt{amp()} to represent a function to retrieve the amplitude of the frequency. 
%
\begin{align*}
&\text{Given input audio }i\text{ and }o\text{, find }t\text{ such that} \\
&\forall f_1 \in  \texttt{spectrogram(i)}.\ \forall f_2 \in \texttt{spectrogram(o)}. \\
&(f_1 > t \land  f_2 > t \land f_1 == f_2) \implies \texttt{amp}(f_1) > \texttt{amp}(f_2)
\end{align*}


Similarly, we can build a formula to describe how to calculate an initial value for a high-pass.
Here we look for the lowest frequency where amplitude has decreased in the output audio - this is then a starting point for the threshold of a high pass filter.
%
\begin{align*}
&\text{Given input audio }i\text{ and }o\text{, find }t\text{ such that} \\
&\forall f_1 \in \texttt{spectrogram(i)}.\ \forall f_2 \in \texttt{spectrogram(o)}. \\
&(f_1 < t \land f_2 < t \land f_1 == f_2) \implies \texttt{amp(}f_1\texttt{)} > \texttt{amp(} f_2 \texttt{)} 
\end{align*}


\subsection{Structural Synthesis}

\begin{figure}
\begin{align*}
	LPF \ 10000 \ 0.5 \arrComp HPF \ 100 \ 0.5 & \\
	\ldots & \arrComp PitchShift \ 0.1 \ 0.1 \\
	\ldots & \parallelCompose PitchShift \ 0.1 \ 0.1 \\
	\ldots & \arrComp Reverb \ 0.1 \ 0 \ 0.1 \\
	\ldots & \parallelCompose Reverb \ 0.1 \ 0 \ 0.1 \\
	\ldots & \arrComp WhiteNoise \ 0.1 \\
	\ldots & \parallelCompose WhiteNoise \ 0.1
\end{align*}
\caption{New structural candidates are generated based on all compositions of unused \dspnode}
\label{fig:generation}
\end{figure}

During each loop of our synthesis procedure, we attempt to build a new structure based on the results of our previous attempts.
We compare several different variations on the current structure (Fig.~\ref{fig:generation}).
Variations are generated by composing all \dspnode that are not present in the current structure with sequential and parallel composition onto the existing filter program returned from our metrical synthesis (cf. Sec.~\ref{sec:opt}).
We keep the parameters for the existing \dspnode, but for the new \dspnode we initialize the parameters to small values.
Using small values (instead of zeroes) ensures that the filter has an observable effect on the sound.
We score each of these filter programs, $f$, by the distance, $\distFxn(f(i), o)$, from their output to the output example file.
We then select the candidate filter program with the best score.
We continue this process until we either exceed our maximum allowed structural attempts or have reached a program whose output is close enough to the desired output.
