
\section{Introduction}

There has been a proliferation of new programming languages for audio Digital Signal Processing (DSP) with languages such as SuperCollider, Faust, MaxMSP, PureData.
These languages provide a high-level interface to make DSP programming more approachable for new-comers to the field of audio programming.
DSP programming poses a particular challenge to novice programmers as writing a DSP program requires an understanding of traditional programming concepts such as loops and conditionals, as well as the ability to reason about DSP issues such as time vs frequency domain.
To assist beginning DSP programmers we turn to the Programming by Example paradigm.

Programming by Example (PBE) is a program synthesis technique that allows users to provide input and output examples to system that then automatically generates code that models the illustrated functionality.
This technique has found particular success in spreadsheets with the FlashFill tool.
In a similar vein, we hope to make DSP programming more accessible to a larger audience by using Digital Signal Processing Programming by Example.


\subsection{Motivating Example}

\markk{I think we should switch this - this is more motivating for epople that don't know what DSP code looks like}

\begin{figure}
\begin{lstlisting}
b = Buffer.read(s, 
  Platform.resourceDir +/+ "x.wav");

SynthDef(\myLPF, {| out = 0, buf |
  Out.ar(out,
    LPF.ar(
      PlayBuf.ar(1, 
                 buf, 
                 BufRateScale.kr(buf)),
      2000)
  )
}).play(s, [\buf, b]);
\end{lstlisting}
\caption{A program in SuperCollider that applies a low-pass filter to an audio buffer}
\label{fig:sc_code}
\end{figure}

Fig.~\ref{fig:sc_code} presents a simple program in SuperCollider.
As difficult as the syntax is in SuperCollider, that is not the most difficult part for beginners.
It is far harder to accomplish the conceptual goal, which is to discover the correct combination of DSP filters to reproduce an audio effect on a new sample.


This paper makes the following contributions:

\begin{enumerate}
\item We propose framework for the synthesis of DSP programs that utilize noncommutative filter types 
\item We propose an algorithm for search through the possible structural forms of a DSP filter program
\item We present our tool, \ourTool, which synthesizes SuperCollider programs from audio input/output examples, and evaluate \ourTool over a set of benchmarks
\end{enumerate}


