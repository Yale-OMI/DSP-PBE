% ===========================================
% Template for ICMC-NYCEMF 2019 (version2)
% adapted from earlier LaTeX paper templates for the ICMC, SMC, etc...
% ===========================================

\documentclass{article}
\usepackage{icmc2019template}
\usepackage{times}
\usepackage{ifpdf}
\usepackage{soul}
\usepackage[english]{babel}
%\usepackage{cite}


\long\def\markk#1{{\color{red}{\bf Mark: }{\small [#1]}}}


%%%%%%%%%%%%%%%%%%%%%%%% Some useful packages %%%%%%%%%%%%%%%%%%%%%%%%%%%%%%%
%%%%%%%%%%%%%%%%%%%%%%%% See related documentation %%%%%%%%%%%%%%%%%%%%%%%%%%
%\usepackage{amsmath} % popular packages from Am. Math. Soc. Please use the 
%\usepackage{amssymb} % related math environments (split, subequation, cases,
%\usepackage{amsfonts}% multline, etc.)
%\usepackage{bm}      % Bold Math package, defines the command \bf{}
%\usepackage{paralist}% extended list environments
%%subfig.sty is the modern replacement for subfigure.sty. However, subfig.sty 
%%requires and automatically loads caption.sty which overrides class handling 
%%of captions. To prevent this problem, preload caption.sty with caption=false 
%\usepackage[caption=false]{caption}
%\usepackage[font=footnotesize]{subfig}

% ====================================================
% ================ Define title and author names here ===============
% ====================================================
%user defined variables
\def\papertitle{Synthesizing DSP Filters from Examples of Analog Effects}
\def\firstauthor{First Author}
\def\secondauthor{Second Author}
\def\thirdauthor{Third Author}
\def\fourthauthor{Fourth Author}
\def\fifthauthor{Fifth Author}
\def\sixthauthor{Sixth Author}

% adds the automatic
% Saves a lot of output space in PDF... after conversion with the distiller
% Delete if you cannot get PS fonts working on your system.

% pdf-tex settings: detect automatically if run by latex or pdflatex
\newif\ifpdf
\ifx\pdfoutput\relax
\else
   \ifcase\pdfoutput
      \pdffalse
   \else
      \pdftrue
  \fi
\fi

\ifpdf % compiling with pdflatex
  \usepackage[pdftex,
    pdftitle={\papertitle},
    pdfauthor={\firstauthor, \secondauthor, \thirdauthor},
    bookmarksnumbered, % use section numbers with bookmarks
    pdfstartview=XYZ % start with zoom=100% instead of full screen; 
                     % especially useful if working with a big screen :-)
   ]{hyperref}
  %\pdfcompresslevel=9

  \usepackage[pdftex]{graphicx}
  % declare the path(s) where your graphic files are and their extensions so 
  %you won't have to specify these with every instance of \includegraphics
  \graphicspath{{./figures/}}
  \DeclareGraphicsExtensions{.pdf,.jpeg,.png}

  \usepackage[figure,table]{hypcap}

\else % compiling with latex
  \usepackage[dvips,
    bookmarksnumbered, % use section numbers with bookmarks
    pdfstartview=XYZ % start with zoom=100% instead of full screen
  ]{hyperref}  % hyperrefs are active in the pdf file after conversion

  \usepackage[dvips]{epsfig,graphicx}
  % declare the path(s) where your graphic files are and their extensions so 
  %you won't have to specify these with every instance of \includegraphics
  \graphicspath{{./figures/}}
  \DeclareGraphicsExtensions{.eps}

  \usepackage[figure,table]{hypcap}
\fi

%setup the hyperref package - make the links black without a surrounding frame
\hypersetup{
    colorlinks,%
    citecolor=black,%
    filecolor=black,%
    linkcolor=black,%
    urlcolor=black
}


% ====================================================
% ================ Title and author info starts here ===============
% ====================================================
% Title.
% ------
\title{\papertitle}

% Authors
% Please note that submissions are anonymous, therefore 
% authors' names should not be VISIBLE in your paper submission.
% They should only be included in the camera-ready version of accepted papers.
% uncomment and use the appropriate section (1, 2 or 3 authors)
%
% Single address
% To use with only one author or several with the same address
% ---------------
%\oneauthor
%   {\firstauthor} {Affiliation \\ %
%     {\tt \href{mailto:author@unt.edu}{author@unt.edu}}}

%Two addresses
% the default spacing is 1.5in, but this can be reduced to 0.5in or less, if needed
%--------------
% \twoauthors
%   {1.5in}
%   {\firstauthor} {Affiliation1 \\  %
%     {\tt \href{mailto:author1@unt.edu}{author1@unt.edu}}}
%   {\secondauthor} {Affiliation2 \\  %
%     {\tt \href{mailto:author2@unt.edu}{author2@unt.edu}}}

% Three addresses
% the default spacing is 0.5in, but this can be reduced to 0.3in or less, if needed
% --------------
 \threeauthors
   {0.5in}
   {\firstauthor} {Affiliation1 \\ %
     {\tt \href{mailto:author1@smcnetwork.org}{author1@smcnetwork.org}}}
   {\secondauthor} {Affiliation2 \\ %
     {\tt \href{mailto:author2@smcnetwork.org}{author2@smcnetwork.org}}}
   {\thirdauthor} { Affiliation3 \\ %
     {\tt \href{mailto:author3@smcnetwork.org}{author3@smcnetwork.org}}}

% Four addresses
% the default spacing is 1.5in, but this can be reduced to 0.5in or less, if needed
% --------------
% \fourauthors
%   {1.5in}
%   {\firstauthor} {Affiliation1 \\ %
%     {\tt \href{mailto:author1@unt.edu}{author1@unt.edu}}}
%   {\secondauthor} {Affiliation2 \\ %
%     {\tt \href{mailto:author2@unt.edu}{author2@unt.edu}}}
%   {\thirdauthor} { Affiliation3 \\ %
%     {\tt \href{mailto:author3@unt.edu}{author3@unt.edu}}}
%   {\fourthauthor} { Affiliation4 \\ %
%     {\tt \href{mailto:author4@unt.edu}{author4@unt.edu}}}

% Five addresses
% the default spacing is 0.5in, but this can be reduced to 0.3in or less, if needed
% --------------
% \fiveauthors
%   {0.5in}
%   {\firstauthor} {Affiliation1 \\ %
%     {\tt \href{mailto:author1@unt.edu}{author1@unt.edu}}}
%   {\secondauthor} {Affiliation2 \\ %
%     {\tt \href{mailto:author2@unt.edu}{author2@unt.edu}}}
%   {\thirdauthor} { Affiliation3 \\ %
%     {\tt \href{mailto:author3@unt.edu}{author3@unt.edu}}}
%   {\fourthauthor} { Affiliation4 \\ %
%     {\tt \href{mailto:author4@unt.edu}{author4@unt.edu}}}
%   {\fifthauthor} { Affiliation5 \\ %
%     {\tt \href{mailto:author5@unt.edu}{author5@unt.edu}}}

% Six addresses
% the default spacing is 0.5in, but this can be reduced to 0.3in or less, if needed
% --------------
% \sixauthors
%   {0.5in}
%   {\firstauthor} {Affiliation1 \\ %
%     {\tt \href{mailto:author1@unt.edu}{author1@unt.edu}}}
%   {\secondauthor} {Affiliation2 \\ %
%     {\tt \href{mailto:author2@unt.edu}{author2@unt.edu}}}
%   {\thirdauthor} { Affiliation3 \\ %
%     {\tt \href{mailto:author3@unt.edu}{author3@unt.edu}}}
%   {\fourthauthor} { Affiliation4 \\ %
%     {\tt \href{mailto:author4@unt.edu}{author4@unt.edu}}}
%   {\fifthauthor} { Affiliation5 \\ %
%     {\tt \href{mailto:author5@unt.edu}{author5@unt.edu}}}
%   {\sixthauthor} { Affiliation6 \\ %
%     {\tt \href{mailto:author6@unt.edu}{author6@unt.edu}}}


% ====================================================
% =============== The document content starts here ===============
% ====================================================
\begin{document}
%
\capstartfalse
\maketitle
\capstarttrue
%
\begin{abstract}
Many computer music languages provide a way to write digital signal processing filters, which can be used to model analog effects such as a trumpet mute, or a voice underwater.
Writing filters that correspond to these effects is a difficult task that requires an deep musical understanding of digital signal processing for audio and domain expertise in the audio programming language of choice.
In order to overcome this challenge, we present a tool, SynthSynth, that allows beginners to construct DSP filters simply by providing example audio files.
We present some preliminary results of applying SynthSynth to the reconstruction of analog effects.
We further explain how we generate code for languages such as SuperCollider and MaxMSP from our internal representation of a DSP filter.
\end{abstract}
%


\section{Introduction}

There has been a proliferation of new programming languages for audio Digital Signal Processing (DSP) with languages such as SuperCollider, Faust, MaxMSP, PureData.
These languages provide a high-level interface to make DSP programming more approachable for new-comers to the field of audio programming.
DSP programming poses a particular challenge to novice programmers as writing a DSP program requires an understanding of traditional programming concepts such as loops and conditionals, as well as the ability to reason about DSP issues such as time vs frequency domain.
To assist beginning DSP programmers we turn to the Programming by Example paradigm.

Programming by Example (PBE) is a program synthesis technique that allows users to provide input and output examples to system that then automatically generates code that models the illustrated functionality.
This technique has found particular success in spreadsheets with the FlashFill tool.
In a similar vein, we hope to make DSP programming more accessible to a larger audience by using Digital Signal Processing Programming by Example (DSP-PBE).

The goal of DSP-PBE is to take an input and output audio example from the user, and synthesize the DSP program, $F$, that minimizes the distance between the transformed input, $F(i)$, and the output $o$.
In this way, a user can automatically generate the program code that captures the audio transformation demonstrated by the provided examples.
A key part of this technique is that the user receives readable DSP program code that can be further tuned or edited as the user sees fit, opening the door for learning opportunities and creative invention.


\subsection{Motivating Example}


\begin{figure}
\begin{lstlisting}
( 
SynthDef(\dsp_pbe, {|out=0|

   var main_in, id7, out6, lpf5, hpf4, psh1;
   main_in = PlayBuf.ar(2, ~buf);
   psh1 = FreqShift.ar(pitchRatio: -399.999, 
                       mul: 0.55, 
                       in: In.ar());
   hpf4 = HPF.ar(freq: 10100.0, 
                 mul: 1.562e-10, 
                 in: psh1);
   lpf5 = LPF.ar(freq: 3860.002, 
                 mul: 0.85, 
                 in: psh1);
   out6 = Mix.ar(2, [hpf4, lpf5]);
   id7 = 0.7 * out6;
   Out.ar(out, id7);

}).add;
)
\end{lstlisting}
\caption{The SuperCollider program synthesized by \ourTool to simulate the effect of a trumpet hat mute.}
\label{fig:sc_code}
\end{figure}

We demonstrate the application of \ourTool by showing how the tool is used to build a SuperCollider program that mimics the effect of a trumpet mute.
A user may want to construct a program in order to apply the effect of a trumpet mute to another sound.
To start, a user would provide an audio input example file of the trumpet without a mute (``00 none'' in Table~\ref{table:eval}), and an audio output example file (``01 hat'' in Table~\ref{table:eval}) of the trumpet playing the same note with the mute.
The user then invokes \ourTool on these two files, and the tool generates the SuperCollider program shown in Fig.~\ref{fig:sc_code}.
With this code, the user can then use the program directly in a larger SuperCollider project.
In addition, the user may wish to edit the code themselves - for example changing the mul argument to the HPF to a larger value.

In summary, this paper makes the following contributions.

\begin{enumerate}
\item We propose framework for the synthesis of DSP programs that utilize noncommutative filter types 
\item We propose an algorithm for search through the possible structural forms of a DSP filter program
\item We present our tool, \ourTool, which synthesizes SuperCollider programs from audio input/output examples, and evaluate \ourTool over a set of benchmarks
\end{enumerate}




\section{System Overview}

How much of this should be the algorithm and how much should be implementation specific

\markk{system image here specialized on analog effects}

\subsection{Aural Distance as a Metric Space}

One of the key components in machine learning systems is having a metric that to quantifies how close the candidate solution is to the desired solution.
In the case of programming-by-example, the desired solution is defined by the user-provided input-output example pairs.
At a high-level, our goal is to minimize the distance, $dist$, between the output of the candidate function, $F$, on the provided input, $I$, and the provided output $O$.
Put mathematically, we have a minimization problem: $min (dist ( F(I)), O)$.

Previous work proposed using a distance metric between two audio files based on constellation plots of multiple FFT plots over the samples~\cite{SantolucitoFARM}.
We adopt the same distance metric for this work.

\subsection{Gradient Descent}

We use a modified version of gradient descent to find good parameters for the filter.
Have to decide how much to go into this - probably too technical for ICMC, but there are few things that are new from FARM that would be nice to highlight

\subsection{Choosing an initial starting point}

As a very rough estimate, if the max freq peak of the output is less than the max freq peak of input
  we need a lpf, and it should have a value a bit less than the max peak of output

\subsection{Searching for DSP structure}

Just brute force for now - maybe guided by user? 
Something smarter can come later.

\subsection{Generating Program Code}

\markk{how to we turn the solution into a PD/MaxMSP/SuperCollider program (see TODO on github, still need to actually code this)}


\section{Evaluation}

To evaluate our approach, we apply our synthesis algorithm to two sets of input/output examples.
The first set captures the use case of \ourTool where a user wants to model real world sounds.
This set consists of trumpet sounds with various mutes.
The second set of example input/output we run \ourTool on focuses on more creative applications.
Here we attempt to transform whitenoise into various clips from Brian Eno's Music for Airports.

The results of running \ourTool on the trumpet mute sounds benchmarks are reported in Table~\ref{table:eval}.

We need to normalize all files as the distance metric is amplitude analysis.

\begin{table}
\begin{tabular}{|l|l|c|c|c|}
\hline
\textbf{Input} & \textbf{Filter} & \textbf{dist(o, $f$(i))} & \textbf{Time (sec)}
\csvreader{results/farm.csv}{}
{\\ \hline \csvcoli & \csvcolii & \csvcoliv & \csvcolvi}
\\ \hline
\end{tabular}
\caption{Demonstrating the accuracy of the user provided refinements for the initial filter.}
\label{table:evalInit}
\end{table}

\begin{table*}[]
\begin{tabular}{|l|l|c|c|c|}
\hline
\textbf{Input} & \textbf{Output} & \textbf{dist(o, $f$(i))} & \textbf{Structural Attempts} & \textbf{Time (sec)}
\csvreader{results/trumpet.csv}{}
{\\ \hline \csvcoli & \csvcolii & \csvcoliv & \csvcolv & \csvcolvi}
\\ \hline
\end{tabular}
\caption{Evaluation on a set of benchmarks.}
\label{table:eval}
\end{table*}






\section{Related}

Previously, the machine learning and semantic modelling approach to program synthesis have been viewed as at odds with one another.
Depending on the context, one or the other outperforms the other~\cite{devlin2017robustfill}.
In this work, we join the two approaches together in a restricted example domain (where the example domain forms a metric space).

The work of~\cite{misailovic2011probabilistically} introduces a notion of approximate correctness of program transformation over probabilistic programs.
Aside from our work focusing on synthesis, we also are working over deterministic programs - our notion of approximate is not with respect to probabilistic outcomes, but closeness in the metric space.



\begin{acknowledgments}
This research sponsored by NSF grants CCF-1302327 and CCF-1715387.
This research also sponsored by \markk{whoever OMI got money from at Yale}
\end{acknowledgments} 

%%%%%%%%%%%%%%%%%%%%%%%%%%%%%%%%%%%%%%%%%%%%%%%%%%%%%%%%%%%%%%%%%%%%%%%%%%%%%
%bibliography here
\bibliography{icmc2019template}

\end{document}
