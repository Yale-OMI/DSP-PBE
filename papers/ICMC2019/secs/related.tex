\section{Related Work}

Various resynthesis techniques - a lot of these are covered in the FARM paper.
\markk{We need a paragraph about physical modelling - how cool it is, but also how difficult it is for beginners. Maybe even say that it is chapter 13 (or whatever - I'm sure it is not one of the first chapters anyway) in that book in AKW123. This is good evidence that we need a different approach to physical modelling that is more accessible to novice DSP programmers. Also not all analog effects can be effectively modelled using physical modelling (e.g. voice underwater - speaking of which, would be nice to have a benchmark audio file for this if anyone wants to search online/do some field recording).}

DSP-PBE is similar to using impulse response for acoustic modelling of rooms, such as studios and concert halls.
Measuring the impulse response allows the user to model how a room affects a source sound.
When measuring the impulse response, we can use any sound we like (typically, a sine wave sweep), and then deconvolve the source signal in order to isolate the frequency response and resonance of the room.
However, in the case of modelling physical, analog effects, the effect can only be applied to a particular sound.

\url{https://www.acoustics-engineering.com/files/TN001.pdf}
\url{https://www.gearslutz.com/board/studio-building-acoustics/469574-measuring-room-acoustics.html}

Programming-by-example is a research field that has had an explosion of growth in the past decade. 
A critical point was the introduction of FlashFill in Microsoft Excel 2013~\cite{GulwaniHS12}, a programming-by-example technology that allows users to automatically generate excel scripts to transform data.
The majority of

In prior work has explored the application of programming-by-example to digital signal processing~\cite{SantolucitoFARM}.
This work was focused on reconstructing DSP filters from digital examples, for example finding the threshold of an applied low-pass filter.
In contrast, this work looks at how programming-by-example can be applied to audio recordings of real-world effects.
One main challenges we face here is that when learning DSP programs to model analog effects, we do not apriori have any information about the correct DSP program.
We must now synthesize, not just the value (e.g. threshold for low-pass), but also the overall structure of the DSP filter.