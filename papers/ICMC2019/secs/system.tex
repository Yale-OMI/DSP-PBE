\section{Background on DSP-PBE}

We give a short introduction on Digital Signal Processing Programming-By-Example (DSP-PBE).
The idea of DSP-PBE was introduced in~\cite{SantolucitoFARM}.

\subsection{Aural Distance as a Metric Space}

One of the key components in machine learning systems is having a metric that to quantifies how close the candidate solution is to the desired solution.
In the case of programming-by-example, the desired solution is defined by the user-provided input-output example pairs.
At a high-level, our goal is to minimize the distance, $dist$, between the output of the candidate function, $F$, on the provided input, $I$, and the provided output $O$.
Put mathematically, we have a minimization problem: $min (dist ( F(I)), O)$.

Previous work proposed using a distance metric between two audio files based on constellation plots of multiple FFT plots over the samples~\cite{SantolucitoFARM}.
We adopt the same distance metric for this work.

\url{https://en.wikipedia.org/wiki/Short-time_Fourier_transform#Sliding_DFT}

\subsection{Gradient Descent}

We use a modified version of gradient descent to find good parameters for the filter.
Have to decide how much to go into this - probably too technical for ICMC, but there are few key points worth highlight (e.g. distance metric needs to be convex-ish).
Should we include a similar convexity graph as in FARM? 


\section{DSP-PBE for analog effects}

In order to use DSP-PBE there are a number of new challenges that we must overcome.
First, we need a more efficient way of choosing an initial point for gradient descent. 
Second, we must have a way to automatically explore different structures of DSP programs. 
Finally, a priority for this system is usability by novice DSP programmers, so we needed our tool to have the ability to generate program code that could be immediately run.

\markk{system image here specialized on analog effects}

\subsection{Choosing an initial point for Gradient Descent}

As a very rough estimate, if the max freq peak of the output is less than the max freq peak of input
  we need a lpf, and it should have a value a bit less than the max peak of output

\subsection{Searching for DSP structure}

\markk{need an image of a DSP program structures (demonstrating both sequential and parallel composition) along with the program code to describe this structure. Ideally made in tikz, but if we run out of time, a screenshot of Max/PD/google slides will work too}

Just brute force for now - maybe guided by user? 
Something smarter can come later.

\subsection{Generating Program Code}

\markk{how to we turn the solution into a PD/MaxMSP/SuperCollider program (see TODO on github, still need to actually code this). Need to be sure to mention that this was not possible in previous FARM paper.}

\samm{added some words on the conversion to supercollider code}

In order to make the filter generated by the DSP-PBE program usable by computer musicians we have implemented a translation scheme to convert the in-program representation of the filter to raw SuperCollider code.