\section{System Overview}

How much of this should be the algorithm and how much should be implementation specific

\markk{system image here specialized on analog effects}

\subsection{Aural Distance as a Metric Space}

One of the key components in machine learning systems is having a metric that to quantifies how close the candidate solution is to the desired solution.
In the case of programming-by-example, the desired solution is defined by the user-provided input-output example pairs.
At a high-level, our goal is to minimize the distance, $dist$, between the output of the candidate function, $F$, on the provided input, $I$, and the provided output $O$.
Put mathematically, we have a minimization problem: $min (dist ( F(I)), O)$.

Previous work proposed using a distance metric between two audio files based on constellation plots of multiple FFT plots over the samples~\cite{SantolucitoFARM}.
We adopt the same distance metric for this work.

\subsection{Gradient Descent}

We use a modified version of gradient descent to find good parameters for the filter.
Have to decide how much to go into this - probably too technical for ICMC, but there are few things that are new from FARM that would be nice to highlight

\subsection{Choosing an initial starting point}

As a very rough estimate, if the max freq peak of the output is less than the max freq peak of input
  we need a lpf, and it should have a value a bit less than the max peak of output

\subsection{Searching for DSP structure}

Just brute force for now - maybe guided by user? 
Something smarter can come later.

\subsection{Generating Program Code}

\markk{how to we turn the solution into a PD/MaxMSP/SuperCollider program (see TODO on github, still need to actually code this)}
