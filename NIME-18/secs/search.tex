\section{Search}

Search techniques for programming by example have been the subject of intense research~\cite{?,?,?}.
As the search space of possible program is extremely large, search procedures must be exceptionally efficient. 

\section{Overview}

The core of our synthesis algorithm operates on a two layer feedback loop that incrementally refines both the possible structure and the constants of the function, as illustrated in Fig.~\ref{fig:overview}.
The algorithm terminates when, for input $i$ and output $o$, the synthesized function maps the input such that $\distFxn(\synthedFilter(i),o)$ is below a user-provided given minimum threshold $\epsilon$.
When metrical synthesis fails to produce a solution that is below the $\epsilon$ threshold, we restart synthesis by searching for a new structure.
After generating a new structure, we apply the metrical values from the previous metrical synthesis attempt to the new structure, and begin metrical synthesis again.

\section{Overview}

The core of our synthesis algorithm operates on a two layer feedback loop that incrementally refines both the possible structure and the constants of the function, as illustrated in Fig.~\ref{fig:overview}.
The algorithm terminates when, for input $i$ and output $o$, the synthesized function maps the input such that $\distFxn(\synthedFilter(i),o)$ is below a user-provided given minimum threshold $\epsilon$.
When metrical synthesis fails to produce a solution that is below the $\epsilon$ threshold, we restart synthesis by searching for a new structure.
After generating a new structure, we apply the metrical values from the previous metrical synthesis attempt to the new structure, and begin metrical synthesis again.

\section{Overview}

The core of our synthesis algorithm operates on a two layer feedback loop that incrementally refines both the possible structure and the constants of the function, as illustrated in Fig.~\ref{fig:overview}.
The algorithm terminates when, for input $i$ and output $o$, the synthesized function maps the input such that $\distFxn(\synthedFilter(i),o)$ is below a user-provided given minimum threshold $\epsilon$.
When metrical synthesis fails to produce a solution that is below the $\epsilon$ threshold, we restart synthesis by searching for a new structure.
After generating a new structure, we apply the metrical values from the previous metrical synthesis attempt to the new structure, and begin metrical synthesis again.

\input{figs/overview}







\subsection{Gradient Descent}

One trouble with this approach is that we have lots of local minimums in the space of optimization.
Take the example of trying to learn a low-pass filter threshold.
If our initial threshold is too low (cutting off too much), and just on the other side of the threshold is a frequency we need, SGD will easily send us in that direction.
However, if the spectrogram of waveform has a frequency spectrum that is mostly inactive, SGD will detect a plateau and tell us we have found a global minimum, when in fact we are stuck in a local minimum.
We may be able to use some randomized restarting to hop over these.

\subsection{Refinement Type Driven Synthesis}

Refinement types are a way of giving an abstract description of the behavior of a function.
For example, we can describe the function \texttt{map:: [a] -> [b]}, that captures some properties of the behvaoir of the function, \texttt{f :: xs:[a] -> ys:[b] | length xs==length ys}, in this case that the length of the lists are still equal.
In a similar style for DSP, we can write predicates about the filters available to us during synthesis.
For example, a low-pass filter could be described a the refinement type that says the amplitude of the frequencies greater than the threshold have decreased in the output \texttt{Audio}.

\texttt{lpf :: threshold:Float -> xs:Audio -> ys:Audio | amp(threshold,xs) < amp(threshold,ys)}




