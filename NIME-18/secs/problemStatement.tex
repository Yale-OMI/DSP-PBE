\section{DSP programming by example}

The problem of DSP programming by example is formally defined as follows:
Given an input waveform $I$ and an output waveform $O$, construct a DSP filter $\synthFilter$, to minimize the aural distance $dist$ between the $O$ and $\synthFilter(I)$.
In a single line:
%
\begin{align*}
\text{Find } \synthFilter, \text{ such that } dist(O,\synthFilter(I))=0
\end{align*}


In the sequel we will describe the two key components of this statement; the definition of distance, and a search technique to find $\synthFilter$.
A distance metric that is faithful to the psycho-acoustics of the human ear is critical for a useful tool.
As an example, taking a trivial distance function that returns the difference in length of the two audio samples will allow a delay filter to satisfy any example pair of samples.

Additionally, an efficient search algorithm is critical, as the space of possible DSP filters is very large.
Not only do we need to consider a wide variety of filters, we need to consider the space of parameters for each filter, as well as the different ways of combining multiple filters.
